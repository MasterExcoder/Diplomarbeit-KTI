\chapter{Einleitung}
\section{Warum?}
Im zunehmenden Maße finden sich mobile Endgeräte (z.B. Tablet) als Marketing, Informations- und Imageelement, aber auch als Teil von Prozessen im Retail- als auch im Industrieumfeld wieder. Diese ersetzen oder erweitern bestehende Lösungen (z.b. Kiosk, Digital-Signage, Laptops, Industrie PCs) um interaktive Elemente und unterstützen so Unternehmen in ihren Geschäftsprozessen. 
Bei der eingesetzten Lösung ist es aus Sicht des Retail und Industrie-Kunden wichtig die Handhabung und das Erlebnis des Gerätes für den Anwender zu erhalten. Die Einsatzgebiete reichen dabei von Tablets in Produktionsprozessen zu Zwecken der Qualitätssicherung (eigene Applikationen mit Dokumentationsfunktionen) bis hin zu Multimedia-Terminals im stationären Handel. Um diese verschiedenen Szenarien realisieren zu können ist es notwendig eine stabile, sichere und wartbare Plattform zu konstruieren, die zum einen die Anwendung in unterschiedlichen Einsatzgebieten ermöglicht und zum anderen für den Betriebsführer leicht bereitzustellen und zu warten ist.\cite{lastenheft}
%Zitat fragwürdig?!



\section{Ursprungsproblem}

\section{Vorgehensweise}
