\chapter{Auswahl \& Konzept}
\section{Einleitung}
\paragraph*{•}
Dieser Untersuchungsbericht dient dazu der Firma Kapsch BusinessCom AG eine Empfehlung für ein bei deren Kunden einzusetzendes Managementsystem für Tablets auf Basis Android auszusprechen. Diese Plattform muss gewissen Vorgaben des Unternehmens entsprechen, welche in der Anfangsphase des Projekts definiert wurden. Nach anfänglichen Untersuchungen konnte des Projektteam vier Kandidaten für Betriebsplattformen in die engere Auswahl ziehen. Diese vier Systeme wurden bezüglich ihres Nutzes für die Absichten der Firma Kapsch untersucht, dokumentiert und in diesem Untersuchungsbericht zusammengefasst. Bei diesen potentiell einsetzbaren Systemen handelt es sich um:
\begin{itemize}
	\item Linux Manipulation
	\item Mobile Device Management
	\item Mobile Device Management + Container Technologien
	\item Samsung Knox
\end{itemize}
Diese werden in diesem Bericht detailliert dargestellt und es wird auf deren spezifische Stärken und Schwächen eingegangen. Am Ende stellt eine Nutzwertanalyse einen Vergleich der Systeme dar, an Hand derer das abschließende Fazit und die damit einhergehende Empfehlung erfolgen. Damit kann die Kapsch BusinessCom AG dann entscheiden, ob und wenn wie sie das System für ihre zukünftigen Projekte einsetzen können. 

\section{Linux Manipulation}
\subsection{Allgemein}
\paragraph*{}
Dieser Teil befasst sich mit der Veränderung des Grundsystems eines jeden Android Geräts. Dieses Grundsystem basiert auf dem Open Source Betriebssystem Linux und existiert dabei in einer für Mobilgeräte angepassten Form. In seiner Standardimplementation bietet es zwar einige Funktion zur Erhöhung der Gerätesicherheit, jedoch nicht genügend, um es in einem betrieblichen Umfeld einzusetzen. Da Linux ein Open Source System ist, darf der Source Code von jedem angesehen und nach eigenen Wünschen verändert werden. Und genau hier setzt die Methode der Linuxmanipulation an. In dem man den Source Code so verändert, dass bestimmte Teile des Betriebssystems unzugänglich gemacht oder verschlüsselt werden, ist es möglich den späteren Benutzer vor unabsichtlichen Änderungen am System zu bewahren, welche den reibungslosen Betrieb stören könnten. Das bedeutet, dass dadurch ein vollkommen an die Bedürfnisse des Kunden angepasstes Betriebssystem möglich wäre. Um sich einen besseren Eindruck davon zu verschaffen, wie diese Manipulationen letztendlich aussehen, lohnt es sich auf die diversen frei am Markt erhältlichen Derivate zu werfen. Bekannte Beispiele dafür wären:
\begin{itemize}
	\item CyanogenMod
	\item Android AOSP
	\item Paranoid Android
	\item Dirty Unicorns
	\item etc.
\end{itemize}
Zwar sind diese nicht mit securitytechnischen Absichten entwickelt worden, aber sie zeigen trotzdem auf was mit einer Menge an Entwicklungsaufwand möglich ist. Da aber durch die tiefgreifenden Eingriffe in das System eines Android Gerätes auch die Garantieansprüche verloren gehen, ist das Projektteam zu einem schnellen Fazit zu kommen.
\subsection{Schlussfolgerung}
Die Methode der Linuxmanipulation ist für die Zwecke der Kapsch leider absolut nicht einsetzbar dar sich bei ihr gewisse Konflikte bezüglich Garantieanspruch und Kosten ergeben. Zwar wären über diese Methode sämtliche Anforderungen an das Endprodukt erfüllbar, jedoch bedarf es dazu eines so großen Entwicklungsaufwands, dass dieser sich in einem dermaßen hohen Endkundenpreis niederschlagen würde, welcher von kaum einem Unternehmen zu bezahlen wäre. Denn nicht nur die Beschäftigung einer Vielzahl von Entwicklern über einen langen Zeitraum hinweg, sondern auch der anfallende Support für das Produkt würde sich selbst für ein großes Unternehmen wie Kapsch nicht rentieren. Ein weiteres großes Manko dieser Methode ist die verfallende Garantie für die Hardware. Egal mit welchen Android Tablet diese Software ausgeliefert werden würde, sobald eine Veränderung der darauf vorinstallierten Software stattfindet, gehen sämtliche Garantieansprüche an den Hersteller verloren. Bei der geplanten Menge an ausgelieferten Geräten durch die Kapsch, wäre dies nicht vertretbar. Für einen industriellen Einsatzzweck ist die Variante daher absolut nicht geeignet und der damit einhergehende Aufwand würde sich nur durch einen extrem hohen Verkaufspreis ausgleichen lassen. Somit bleibt dem Projektteam nichts anderes zu sagen, als, dass diese Variante nicht passend für die Absichten der Kapsch ist.

\section{Mobile Device Management (MDM)}
\subsection{Allgemein}
Die folgenden Zeilen beschäftigen sich mit den Einsatz von Mobile Device Management Systemen als Betriebsplattform für potentielle Kunden der Firma Kapsch. Durchgeführt wurden alle Untersuchungen am bereits  am Markt etablierten MDM-System MobileIron. Es wird beleuchtet was der MDM-Standard ist, welche Funktionen für den alltäglichen Gebrauch unumgänglich sind und wie diese im System realisiert sind. Ein besonders wichtiger Punkt hierbei ist auch das Aufzeigen von nicht vorhandenen Funktionen, die jedoch für das Unternehmen Kapsch von fundamentaler Wichtigkeit sind. Des Weiteren wird auf die Bedienbarkeit und die Komplexität der Installation eingegangen und inwiefern dies für den Projektauftraggeber und dessen Kunden relevant ist. Abschließend  wird ein Statement abgegeben, ob bzw. wie es möglich ist diese Form von System für die von Kapsch gedachten Zwecke einzusetzen.
\subsection{Mobile Device Management Standard}
MDM bezeichnet einen von der Open Mobile Alliance (OMA) festgelegten industriellen Standard zur Verwaltung mobiler Endgeräte wie zum Beispiel Smartphones, Tablets oder Laptops. Es dient dazu die allgemeine Verwaltung einer Vielzahl von Geräten zu erleichtern und somit Zeit und Kosten zu sparen. Die mobile Hardware kann dabei vom Unternehmen zur Verfügung gestellt werden oder, sofern mit dem Mitarbeiter abgesprochen, von diesem selbst mitgebracht werden. „Bring Your Own Device“ (BYOD) nennt sich dieser Ansatz. Dieser Standard wird in der Software verschiedenster Hersteller implementiert, welche dann dieses Komplettsystem verschiedenen Unternehmen zur Verwaltung ihrer Geräte anbieten. Beispiele dafür sind.
\begin{itemize}
	\item MobileIron
	\item Samsung EMM
	\item Cisco Meraki
	\item MaaS360
	\item AirWatch
	\item etc.
\end{itemize}
Bestandteile dieser Softwarelösung sind eine Serverkomponente und die verschiedenen mobilen Clients. Der Server dient dabei dazu die Konfigurationen und Statistiken für die Geräte zu halten und zu verwalten. Er sendet auch, auf dem MDM-Standard basierende, Management Kommandos an die Clients aus, wenn sich ein Parameter in deren Konfiguration verändert hat. War es am Anfang der MDM-Systeme noch notwendig das Gerät dazu physisch mit dem Server zu verbinden, geschieht dies heute vollautomatisch über Netzwerkverbindungen. Die implementierten Funktionen können zum Beispiel eine over-the-air (OTA) Verteilung von Applikationen, Daten oder Konfigurationen sein. So braucht ein Administrator nicht auf 100 Geräten das Wifi-Netzwerk einrichten, sondern kann per Knopfdruck diese Konfiguration auf alle im System registrierten Geräte verteilen.  Auch in Punkto Sicherheit bieten MDM-Systeme einige Möglichkeiten und deshalb sind sie so interessant für die Zukunftspläne der Firma Kapsch. So bieten diese Systeme die Möglichkeit Passwortrichtlinien zu setzen oder sogar ganze Teile des Betriebssystems zu sperren, damit diese für den Benutzer nicht zugänglich sind. Dadurch soll die Anfälligkeit für Fehler im Berufsumfeld gesenkt und ein ordentlicher Arbeitsablauf genehmigt werden.
\subsection{Informationen der getesteten Software}
\begin{table}[h]
\centering
\begin{tabular}{|l|l|}
\hline
\textbf{Name}          & MobileIron EMM                                                                                           \\ \hline
\textbf{Hersteller}    & \begin{tabular}[c]{@{}l@{}}MobileIron, 415 East\\ Middlefield Road, Mountain View, CA 94043\end{tabular} \\ \hline
\textbf{Version}       & 7.5.0                                                                                                    \\ \hline
\textbf{Datum}         & 27.01.2015                                                                                               \\ \hline
\textbf{Preis}         & /                                                                                                        \\ \hline
\textbf{Website}       & http://www.mobileiron.com/                                                                               \\ \hline
\textbf{Dokumentation} & https://support.mobileiron.com/eval/                                                                     \\ \hline
\end{tabular}
\caption{MobileIron Übersicht}
\end{table}
\subsection{Installation}
Die Installation von MobileIron gestaltet sich relativ einfach, wobei doch einige wichtige Dinge zu beachten sind. Nach dem Download einer Datei aus dem Online-Zugangsportal von MobileIron, kann über diese das Betriebssystem installiert werden. Diese gestaltet sich für erfahrene Nutzer sehr einfach, allerdings ist auf folgende Dinge acht zu geben: \newline
Folgende Daten müssen bereit stehen bzw. eingerichtet werden:
\begin{itemize}
	\item Lizensierungsinformationen (Firma, Kontaktperson, E-Mail)
	\item IP-Adresse
	\item Externer Hostname \textbf{(Sehr wichtig, weil die mobilen Geräte den Server von außerhalb erreichen müssen)}
	\item Command Line Interface Passwort
	\item Administratorname und -passwort
	\item Mind. 1 physikalisches Interface
	\item Subnetzmaske
	\item Default Gateway
	\item Mind. 1 zu erreichender DNS-Server
	\item Wahlweise
	\begin{itemize}
		\item SSH-Zugriff
		\item Telnet Zugriff
		\item NTP
	\end{itemize}
\end{itemize}
Ist die Einrichtung erfolgt, kann man das System nach einem Neustart bereits einsetzen. Während dem Evaluierungsprozess sind dem Projektteam allerdings einige wichtige Details aufgefallen. Ein funktionierender externer Hostname ist von höchster Wichtigkeit, weil ohne ihn zwar die Einrichtung der Software auf den mobilen Endgeräten funktioniert, leider jedoch die Verbreitung von Konfigurationen versagt. Da jedoch 99 Prozent aller modernen Unternehmen über solche Möglichkeiten verfügen sollten, dürfte dies im realen Betrieb weniger problematisch ausfallen. Hervorzuheben ist hierbei die hervorragende Dokumentation, die MobileIron für den Installationsprozess zur Verfügung stellt. Auf deren Website findet sich eine Sammlung an Dokumenten, welche den Administrator am Anfang zwar überwältigen könnten, aber sich als eine schnell zu durchforstende Sammlung an bebilderten Skripten zur Einrichtung sämtlicher Funktionen herausstellen. Generell lässt sich die Webplattform, welche MobileIron hier zur Verfügung stellt, gut bedienen und bietet Informationen zu den verschiedenen Implementierungsszenarien und Komponenten des Systems. So findet man sich nach kurzer Zeit bereits relativ gut zurecht und weiß wo man suchen muss, um die benötigte Information zu finden.