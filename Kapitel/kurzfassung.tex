\chapter*{Kurzfassung}
\section*{Aufgabenstellung}
Die Aufgabe von Kapsch an das Projektteam ist es, verschiedene Softwarelösungen zum Systemschutz von Android-Tablets zu testen. Dazu soll ein Untersuchungsbericht angefertigt werden. In diesem werden die Ergebnisse  festgehalten und verglichen, um das beste System für die Kunden von Kapsch zu ermitteln. Ein Prototyp in Form eines Android-Tablets ist ebenfalls zu erstellen. Dieser soll mit dem besten, aus dem Untersuchungsbericht hervorgegangenen, Schutzsystem ausgestattet sein.
\section*{Realisierung}
Die Firma Kapsch wünscht einen Untersuchungsbericht, aus dem hervorgeht, welcher Systemschutz für Android-Tablets am besten für spezifische Kunden geeignet ist. In diesem Bericht werden 2-3 verschiedene Systeme zur Absicherung eines Android-Tablets verglichen. Darin wird auf die Vor- und Nachteile des jeweiligen Systems eingegangen, sowie auf das Fehlen von Funktionen aufmerksam gemacht. Dies ermöglicht dem Auftraggeber das bestmögliche System für seine Kunden zu ermitteln.
\section*{Ergebnisse}
Neben dem Untersuchungsbericht wird zusätzlich ein Tablet als Prototyp abgeliefert. Dieses Tablet ist mit dem Schutzsystem ausgestattet, welches aus dem Untersuchungsbericht als am empfehlenswertesten hervorgeht.
\newline Am Ende des Projekts stehen 2 Endprodukte. 
\subsection*{Der Untersuchungsbericht}
Er vergleicht 2 bis 3 Softwarelösungen zur Absicherung von Tablets. Pro Lösung müssen ihre Vor- und Nachteile vorhanden sein. Sowie auch das Fehlen von Einstellungsmöglichkeiten. 

Am Ende des Untersuchungsberichts hat ein Vergleich der Systeme zu stehen, aus dem herausgeht, welches, nach der Meinung des Projektteams, eingesetzt werden sollte. Dies muss mit Argumenten bekräftigt werden.
\newpage
\subsection*{Der Prototyp}
Ein Android-Tablet welches auf die Anforderungen eines Beispielunternehmens zugeschnitten ist. Dabei sollen nur bestimmte Aktionen möglich sein. 
\begin{itemize}
	\item Nutzung von 3 bestimmten Apps
	\item Kein Zugang zu den Einstellungen
	\item Keine Verbindung mit einem Computer möglich
	\item Kein Download von Apps, Fotos, Videos, etc.
	\item Kein Verlassen der gesicherten Umgebung
\end{itemize}

\newpage

\chapter*{Abstract}
\section*{Assignment of Tasks}
Our project partner Kapsch provides enterprise grade solutions for major operating systems such as Microsoft Windows 8. Due to the Consumerization trend in industry platforms like Apple iOS and Android OS are on the rise in enterprise applications and leading system integrators like Kapsch are in need to understand those platforms, their applications and limitations. Based on experience and results from existing projects Kapsch has realized that for certain applications or scenarios the Apple iOS platform has its limitations and drawbacks. In particular industrial applications have a need for a rock-solid platform which enables a system integrator like Kapsch to operate a 24/7 application and service. So Kapsch is keen to expand their knowledge and experience towards the Android platform to be able to provide solution platform for industrial applications.
Aim of this cooperation is to co-work on a set of different feasibility concepts for realizing such a platform based on Android.
\section*{Realisation}
The company Kapsch wants an investigation report, which shows which system protection is suitable for Android tablets best for specific customers. In this report, 2-3 different systems to secure a Android tablets are compared. It takes into account the advantages and disadvantages of each system, as well as draw attention to the lack of features. This allows the customer the best possible system for its customers to determine.
\section*{Results}
In addition to the investigation report a tablet is delivered as a prototype. This tablet is equipped with the protection system, which is apparent from the investigation report as most recommendable.
\newline At the end of the project, there are 2 final results:
\subsection*{The Investigation Report}
It compares 2 to 3 software solutions for securing tablets. For each Solution, advantages and disadvantages must be present. As well as the lack of configuration options.

At the end of the investigation report, a comparison of the systems has to stand, out of the proceeds, which should be in accordance with the opinion of the project team used. This must be confirmed with arguments.
\newpage
\subsection*{The Prototype}
An Android tablet that is tailored to the requirements of a model company. It should be possible only certain actions.
\begin{itemize}
	\item Use of 3 specific apps
	\item No access to the settings
	\item No connection to a computer possible
	\item No downloading of apps, photos, videos, etc...
	\item Leaving the secure environment not possible
\end{itemize}
