\chapter{Lessons Learned}
\section{Sebastian Götze}
Durch das Diplomprojekt Kapsch Tablet Infrastructure habe ich gelernt, dass es besonders bei der Recherche wichtig ist, sich nicht immer auf den Inhalt von Dokumenten und Präsentationen zu verlassen, weil diese oft nicht die Realität widerspiegeln. Besonders beim System Samsung Knox hat sich dies deutlich herauskristallisiert, da in Samsungs Unterlagen zwar steht, dass es einen Modus gibt, in dem der Nutzer sich ausschließlich in einem abgesicherten Container befindet, aber dieser in der Praxis nur bedingt existiert. Der Modus war vorhanden, allerdings war es für den Nutzer jederzeit möglich aus diesem herauszuspringen. Ein weiterer wichtiger Lernaspekt für mich war, dass man bei der Installation von unbekannten Systemen vor Beginn die Installationsanleitung lesen sollte und sich nicht danach auf Fehlersuche zu begeben. Besonders bei der Einrichtung der zusätzlichen Containerapplikationen zu MobileIron wäre dies von Vorteil gewesen, da es hier einige wichtige Details zu beachten gab, ohne die der Prozess nicht abgeschlossen werden konnte.
\section{Samuel Hammer}
Während das Projekts KTI habe ich in erster Linie vieles bezüglich Projektmanagement gelernt. Durch die oftmals eng gesetzten Zeitintervalle war es äußerst wichtig, die Koordination zwischen den zwei Subteams bzw. den einzelnen Teammitgliedern zu perfektionieren. 
Ein teilweise großes Problem war auch, dass sich projektbezogene Abgaben und Termine immer wieder mit Tests, Schularbeiten und anderen unterrichtsbezogenen Verpflichtungen überschnitten. Hier war es von großer Wichtigkeit, sich mit den zuständigen Lehrern bzw. dem Projektpartner abzusprechen. 
Einiges gelernt habe ich natürlich auch in Hinblick auf die wichtigen Dokumente, welche es im Laufe des Projektes zu erstellen und aktualisieren gab.
Alles in Allem war das Projekt "Kapsch Tablet Infrastructure" eine wertvolle Erfahrung, aus der ich viel für meine zukünftige berufliche Laufbahn mitnehmen werde.

\newpage

\section{Michael Kaufmann}
Einer der wichtigsten Punkte, die ich durch die Arbeit am Diplomprojekt in Erfahrung bringen konnte, ist, wie wichtig gut funktionierende Kommunikation innerhalb des Projektteams ist. Um den Arbeitsaufwand bestmöglich zu verteilen und um, so produktiv wie möglich, sein zu können haben wir Sub-Teams gebildet und unser Projekt in einen technischen und einen dokumentationsbezogenen Teil geteilt.
Das Problem hierbei war nur, dass die Sub-Teams teilweise abhängig von einander waren und es hin und wieder vorkam, dass die Arbeit, auf die das eine Team warten musste, um weiter machen zu können, bereits erledigt war, aber nichts davon wussten. Dies führte logischerweise zu Verzögerung des gesamten Projekts. 
Außerdem habe ich gelernt, dass man der Dokumentation und den Whitepapers einer Fertiglösung nicht immer blind vertrauen kann. Denn, wie sich zum Beispiel im Falle Samsung Knox herausgestellt hat, werden oft Funktionen versprochen, welche tatsächlich nicht existieren. Dies führt nicht nur zu einer Verzögerung des Aufwandes, was das Testen der Lösung betrifft, sondern auch dazu, dass die Lösung vor dem Testen besser eingeschätzt wurde, als sie tatsächlich war.

\section{Konstanze Müller}
Das diesjährige Diplomprojekt mit unserem Partner der Kapsch BusinessCom AG gewährte mir in erster Linie einen Einblick in einen kompletten Projektdurchlauf in der Praxis. Theoretische Inhalte für einen reibungslosen Projektverlauf, welche wir in den Jahren davor im Unterricht gelehrt bekommen haben, konnten wir als Projektteam in diesem Abschlussjahr erfolgreich umsetzen. Auch wenn unser Projektleiter Philip Steinhäuser in diesem Bereich die Oberhand hatte, unterstützte ich ihn so gut wie möglich und war somit auch mit formalen Abwicklungskriterien vertraut. 
Aus projekttechnischer Sicht lernte ich, dass man den Aufwand für die Evaluierung von einem System schwer im Voraus abschätzen kann. Des Öfteren war es der Fall, dass Anforderungen von der Checkliste nicht auf Anhieb funktionierten. Selbstverständlich hört man nicht gleich auf zu testen, sondern probiert weiteres. Das Schwierige meiner Meinung war nun den richtigen Zeitpunkt zu finden, um mit einer hohen Wahrscheinlichkeit sagen zu können, dass diese Anforderung vom ausgewählten System nicht unterstützt wird. In Summe bedeutet dies, dass man in Projekten wie diesem sich selbst eine Grenze stecken muss, um auch ein schlussendliches Ergebnis vorlegen zu können. 

\section{Philip Steinhäuser}
Für mich als Projektleiter hat dieses Projekt ein Menge an Erfahrungen nicht nur im Managementbereich, sondern auch im zwischenmenschlichen Umgang bereitgehalten, da ich in einigen Situationen dazu gezwungen war, ein Machtwort zu sprechen, wenn es Streitigkeiten innerhalb des Projektteams gegeben hat. Zusätzlich habe ich einiges zum Thema Zeitplanung und Terminkoordination gelernt, was mir auf meinem weiteren Weg, welcher mich hoffentlich in die Wirtschaft bzw. einen wirtschaftlichen Beruf führen wird, helfen wird. Zu guter Letzt bleibt mir nicht mehr zu sagen, als dass ich dieses Projekt als ein für mich sehr erfahrungsreiches Ereignis verbuche, welches mir in den Bereichen der Teamführung bzw. Projektleitung oder Projektcontrolling sowie Projektmanagement als auch Krisenmanagement eine bzw. mehrere Lehrstunden erteilt hat. Dennoch möchte ich nichts was wir an dem Projekt gemacht haben sowie die Zusammenarbeit mit diesem großartigen Projektteam missen.